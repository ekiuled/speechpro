\documentclass[12pt,a4paper]{article}
\usepackage[russian]{babel}
\usepackage{fontspec}
\setmainfont[Mapping=tex-text]{CMU Serif} 
\setmonofont[Mapping=tex-text]{CMU Typewriter Text}
\usepackage{amsmath}
\usepackage{indentfirst}
\usepackage{fullpage}

\begin{document}

\section*{1}
Воспользуемся методом Якоби для решения уравнения $Ax=b$, который сходится, если $A$ имеет диагональное преобладание. $A=D+L+U$, где $L$ --- нижняя треугольная часть, $U$ --- верхняя треугольная часть, $D$ --- диагональ. 

$(D+L+U)x=b \Rightarrow Dx=-(L+U)x+b \Rightarrow x=D^{-1}(b-(L+U)x)$. 

Положим $x_{k+1}=D^{-1}(b-(L+U)x_k)$. Переписывая в виде $x_{k+1}=Bx_k+q$, получаем:

$B=-D^{-1}(L+U)$, 
$q=D^{-1}b=\begin{pmatrix}
    \frac{b_1}{A_{11}} \\
    ...\\
    \frac{b_n}{A_{nn}}
\end{pmatrix}$,
$Bx_k=-D^{-1}(L+U)x_k=-\begin{pmatrix}
    \frac{1}{A_{11}}\sum\limits_{j\not=1} A_{1j}(x_k)_j \\
    ... \\
    \frac{1}{A_{nn}}\sum\limits_{j\not=n} A_{nj}(x_k)_j
\end{pmatrix}$.

\section*{2}
$x_{k+1}=Bx_k$

Если $||B||<1$, $x_k \rightarrow$ \textbf{0} (все компоненты предела одинаковы).
Если $||B||>0$, $x_k$ расходится. 
Если $B=I$, $x_k \rightarrow x_0$, но для произвольной $||B||=1$ нет сходимости.

\end{document}
